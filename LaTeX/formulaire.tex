\documentclass[a4paper,10pt,multicol]{article}

\usepackage[utf8]{inputenc}
\usepackage[francais]{babel}
\usepackage[T1]{fontenc}

\usepackage{lscape}
\usepackage{multicol}
%\usepackage[small,compact]{titlesec}
\usepackage{times}
\usepackage{graphicx}
\usepackage{tikz}
\usetikzlibrary{fit,calc,positioning,decorations.pathreplacing,matrix}
\usepackage{amsmath,amsfonts,amssymb}
\usepackage{nopageno}


\usepackage{color}
\definecolor{lightgray}{gray}{.85}
\definecolor{darkgray}{gray}{.7}

\usepackage{geometry}
\geometry{hmargin=0.1cm,vmargin=0.2cm}
\setlength{\parindent}{0cm}

\newcommand{\p}{\rangle }
\newcommand{\q}{\langle } 

\paperheight=297mm
\paperwidth=210mm

\setlength\columnseprule{.1pt}
\setlength{\columnsep}{1.4pc}

% SquishList
\newcommand{\squishlist}{
   \begin{list}{$\cdot$}
    { \setlength{\itemsep}{0pt}    \setlength{\parsep}{0pt}
      \setlength{\topsep}{0pt}     \setlength{\partopsep}{0pt}
      \setlength{\leftmargin}{1em} \setlength{\labelwidth}{1.1em}
      \setlength{\labelsep}{0.5em} }
}

\newcommand{\squishend}{
    \end{list}
}

\newcommand{\ud}{\textrm{d}}

% text notation for operators
%\newcommand{\ugrad}{\ensuremath{\overrightarrow{\textrm{grad}} \ }}
%\newcommand{\udiv}{\ensuremath{\textrm{div} \ }}
%\newcommand{\urot}{\ensuremath{\overrightarrow{\textrm{rot}} \ }}

% nabla noation for operators
\newcommand{\ugrad}{\nabla}
\newcommand{\udiv}{\nabla \cdot}
\newcommand{\urot}{\nabla \wedge}

\newcommand{\ulap}{\ensuremath{\nabla^2}}

\newcommand{\tq}{\ensuremath{\ \textrm{tq.} \ }}
\def \dd  {{\rm d}}

\newcommand{\der}[2]{\ensuremath{\frac{\ud #2}{\ud #1}}}
\newcommand{\dder}[2]{\ensuremath{\frac{\ud^2 #2}{\ud #1^2}}}
\newcommand{\D}[2]{\ensuremath{\frac{\partial #2}{\partial #1}}}
\newcommand{\DD}[2]{\ensuremath{\frac{\partial^2 #2}{\partial #1^2}}}

\renewcommand{\vec}[1]{\mbox{\boldmath$#1$}}
\newcommand{\univec}[1]{\ensuremath{\hat{\vec #1}}}
\newcommand{\scalar}[2]{\ensuremath{\vec{#1} \cdot \vec{#2}}}
\newcommand{\cross}[2]{\ensuremath{\vec{#1} \wedge \vec{#2}}}

\newcommand{\emc}[1]{\ensuremath{\frac{#1}{4 \pi \varepsilon_0}}}

\newcommand{\N}{\hat{\vec n}}
\newcommand{\half}{\frac{1}{2}}

\newcommand{\graysection}[1]{\section{\colorbox{lightgray}{#1}}}
\newcommand{\graysec}[1]{\noindent\colorbox{darkgray}{\makebox[\columnwidth][l]{\textbf{#1}}}\\}
\newcommand{\graypar}[1]{\noindent\colorbox{lightgray}{\makebox[\columnwidth][l]{\textbf{#1}}}\\}

\usepackage{amssymb}

\usepackage{pdflscape}

\begin{document}
\begin{landscape}
\begin{multicols}{4}
\raggedcolumns

\setcounter{unbalance}{100}

\pagestyle{empty}


\scriptsize
%\footnotesize

% ---------------------------------
%  DEBUT DU CONTENU DU FORMULAIRE
% ---------------------------------

\graysec{Mathématique}

\graypar{Formules trigonométriques}

\squishlist
\item $\cos(\alpha \pm \beta) = \cos\alpha \cos \beta \mp \sin\alpha \sin\beta$
\item $\sin(\alpha \pm \beta) = \sin\alpha \cos \beta \pm \sin\beta \cos\alpha$
\item $\tan(\alpha \pm \beta) = \frac{\tan\alpha \pm \tan\beta}{1 \mp \tan\alpha\tan\beta}$
\item $\sin(2\alpha) = 2\sin\alpha\cos\alpha$
\item $\cos(2\alpha) = \cos^{2}\alpha - \sin^{2}\alpha = 1-2\sin^{2}\alpha = 2\cos^{2}\alpha$
\item $\frac{1 + \cos\alpha}{2} = \cos^{2}(\alpha/2)$ ; $\frac{1-\cos\alpha}{2} = \sin^{2}(\alpha/2)$
\item $\cos\alpha\cos\beta = 1/2 (\cos(\alpha+\beta) + \cos(\alpha-\beta))$
\item $\cos\alpha\sin\beta = 1/2 (\sin(\alpha+\beta) - \sin(\alpha-\beta))$
\item $\sin\alpha\sin\beta = 1/2 (\cos(\alpha-\beta) - \cos(\alpha+\beta))$
\squishend

\graypar{Intégrales/Dérivées de fonctions trigonométriques}

\squishlist
\item $\int \sin^{2}(ax) \dd x = \frac{x}{2} - \frac{\sin(2ax)}{4a}$
\item $\int \cos^{2}(ax) \dd x = \frac{x}{2} + \frac{\sin(2ax)}{4a}$
\item $\frac{\dd}{\dd x} \tanh(x) = \frac{1}{\cosh^{2}(x)} = {\rm sech}^{2}(x)$ (valable pour $\tan$)
\squishend

\graypar{Développement limité des fonctions trigo. autour de $0$}
\squishlist
\item $\sin(x) \simeq x - \frac{x^{3}}{6}$ \, ; \, $\cos(x) \simeq 1 - \frac{x^{2}}{2}$ \, ; \, $\tan(x) \simeq x + \frac{x^{3}}{3}$
\item $\cot(x) \simeq \frac{1}{x} - \frac{x}{3} - \frac{x^{3}}{45}$  \, ; \, $\sec(x) \simeq 1 + \frac{x^{2}}{2} + \frac{5x^{4}}{24}$
\item $\sinh(x) \simeq x + \frac{x^{3}}{6}$  ; \, $\cosh(x) \simeq 1 + \frac{x^{2}}{2}$  ; \, $\tanh(x) \simeq x - \frac{x^{3}}{3}$
\item $\coth(x) \simeq \frac{1}{x} + \frac{x}{3} - \frac{x^{3}}{45}$ \, ; \, ${\rm sech}(x) \simeq 1 - \frac{x^{2}}{2} + \frac{5x^{4}}{24}$
\squishend

\graypar{Nombres complexes}
\squishlist
\item $\cos(x) = \frac{e^{ix}+e^{-ix}}{2}$ \,;\, $\sin(x) = \frac{e^{ix}-e^{-ix}}{2i}$
\item $\cosh(x) = \frac{e^{x}+e^{-x}}{2}$ \,;\, $\sinh(x) = \frac{e^{x}-e^{-x}}{2}$
\item $e^{i(\pi + 2k\pi)} = -1$ \,; \, $e^{i\cdot 2k\pi} = 1$ \,\, $k \in \mathbb{Z}$
\item $e^{i(\frac{\pi}{2} + 2k\pi)} = i$ \,; \, $e^{i(-\frac{\pi}{2} + 2k\pi)} = -i$ \,\, $k \in \mathbb{Z}$  
\item $\sqrt{2i} = 1+i$\, ; \, $\frac{1}{i} = -i$\, ; \,
\squishend

\graypar{Autres}
\squishlist
\item  $\int e^{-(x-\alpha)^2} \dd x = \sqrt{\pi}$ ; $\int e^{-\alpha x^2} \dd x=\sqrt{\frac{\pi}{\alpha}}$ ; $\int x^2 e^{-\alpha x^2} \dd x = \frac{\sqrt{\pi}}{2\alpha^{3/2}} $ ; $\int e^{-x^2+\alpha x} \dd x= \sqrt{\pi} e^{\alpha^2 /4}$
\item $(x+y)^n=\sum_k^n\begin{pmatrix} n \\ k \end{pmatrix}x^{n-k}y^k$ \, , \, $\begin{pmatrix} n \\ k \end{pmatrix}=\frac{n!}{k!(n-k)!}$
\item $e^x=\sum \dfrac{x^n}{n!}$ ; $\sum\limits_{j=0}^{n-1}x^j=\frac{x^n-1}{x-1}$ ; $\sum\limits_{j=1}^{3}x^j=\frac{x(x^3-1)}{x-1}$ (! $x\neq 1$)
\squishend

\graysec{Nouvelle section}

\graypar{Sous-partie}
\squishlist
\item Premier élément de la liste:
\squishlist
\item Premier élément de la sous-liste
\item Deuxième élément de la sous-liste
\squishend
\item Deuxième élément de la liste
\squishend


% ---------------------------------
%  FIN DU CONTENU DU FORMULAIRE
% ---------------------------------

\end{multicols}
\end{landscape}
\end{document}